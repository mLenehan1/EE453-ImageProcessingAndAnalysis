\documentclass[a4paper]{article}

\usepackage[utf8]{inputenc}
\usepackage[T1]{fontenc}
\usepackage{textcomp}
\usepackage[UKenglish]{babel}
\usepackage{amsmath, amssymb}

\setlength{\parindent}{0pt}
\setlength{\parskip}{1em}


% figure support
\usepackage{import}
\usepackage{xifthen}
\pdfminorversion=7
\usepackage{pdfpages}
\usepackage{transparent}
\newcommand{\incfig}[1]{%
	\def\svgwidth{\columnwidth}
	\import{./figures/}{#1.pdf_tex}
}

\pdfsuppresswarningpagegroup=1

\begin{document}
	\section{Part 1: Thresholding}
	\subsection{Introduction}
	Thresholding is a technique used to segment an image based on grey level
	intensities within the image. Two common methods of thresholding an
	image are applying a "Fixed Global Threshold" and applying an "Adaptive
	Threshold". Each of these methods take a greyscale input image, and
	output a binary (black and white) image.
	\subsection{Techniques}
	\par Fixed Global thresholding involves applying a single threshold value
	across the image, i.e. if the intensity value of the pixel is greater
	than the threshold value, set that pixel to white, otherwise, set it to
	black.
	\par Adaptive thresholding techniques base their threshold values at the
	current pixel off the neighbouring pixels. The "5x5 Adaptive
	Thresholding" method utilised in this section takes the mean intensity
	value of the 24 pixels surrounding the currently selected pixel as the
	threshold value for the currently selected pixel. If this center pixels
	intensity value is greater than the threshold, it's value is set to
	white, otherwise it is set to black.
	\subsection{Pseudocode}
	\subsubsection{Part a}
	\begin{enumerate}
		\item Load the image into Matlab using the ``imread()''
			function
		\item Use the ``rgb2gray()'' function to convert the image to
			greyscale.
		\item Display the greyscale image using the ``imshow()''
			function.
	\end{enumerate}
	\subsubsection{Part b}
	\begin{enumerate}
		\item Follow the steps of Part a.
		\item Apply a fixed global threshold using the VSG package
			'Threshold' function. An arbitrary value should be
			chosen for the fixed threshold value.
		\item Vary the fixed threshold value in order to obtain optimal
			background/foreground segmentation, while retaining
			facial features.
		\item Display the thresholded images using the ``imshow()''
			function.
	\end{enumerate}
	\subsubsection{Part c}
	\begin{enumerate}
		\item Follow the steps of Part a.
		\item Apply a 5x5 adaptive threshold using the VSG package
			"5x5Thresh" function.
		\item Display the adaptive thresholded image using the
			``imshow()'' function.
	\end{enumerate}
	\subsubsection{Part d}
	\begin{enumerate}
		\item Follow the steps of Part a.
		\item Apply Gaussian noise to the greyscale image using the MIP
			``imnoise()'' function. Zero-mean noise should be used
			for this section.
		\item Execute the fixed global thresholding and 5x5 adaptive
			thresholding procedures as described in part b and part
			c respectively.
		\item Adjust the variance value of the ``imnoise()'' function.
			Repeat steps 2 and 3.
	\end{enumerate}
	\subsection{Results}
	\subsection{Conclusion}
	\section{Part 2: Segmentation}
	\subsection{Introduction}
	\subsection{Techniques}
	\subsection{Pseudocode}
	\subsubsection{Part a}
	\subsubsection{Part b}
	\subsection{Results}
	\subsection{Conclusion}
	\section{Part 3: Convolution}
	\subsection{Introduction}
	\subsection{Techniques}
	\subsection{Pseudocode}
	\subsection{Results}
	\subsubsection{Part a}
	\subsubsection{Part b}
	\subsubsection{Part c}
	\subsection{Conclusion}
	\section{Appendix}
\end{document}
